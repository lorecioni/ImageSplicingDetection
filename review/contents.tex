%-- Intro --%

\begin{tframe}{Introduction}

\vspace{0.5cm}
Digital images are easy to manipulate thanks to the availability of the \textbf{powerful editing software} and \textbf{sophisticated digital cameras}.

\vspace{1cm}


\begin{minipage}{\textwidth}
\begin{columns}[T]
\begin{column}{0.5\textwidth}
\vspace{0.1cm}
The development of methods for verifying \textbf{image authenticity} is a real need in forensics.

\vspace{0.8cm}
\textbf{Purpose}: to detect image splicing  aimed at \emph{deceiving} the viewer.
\end{column}
\begin{column}{0.4\textwidth}
\includegraphics[width=0.8\textwidth]{images/image-editing.jpg}
\end{column}
\end{columns}
\end{minipage}

\end{tframe}

%-- Image compositions --%

\begin{tframe}{Forgery detection}
\vspace{0.2cm}
Image splicing detection techniques are based on \textit{inconsistencies}:
\vspace{0.3cm}
\begin{enumerate}
\item \textbf{Image resampling, copy-paste}: deduced from image metadata.
\vspace{0.3cm}

\item \textbf{Compression-based inconsistencies}: JPEG compression introduces blocking artifacts. Manufacturers of digital cameras and image processing software typically use different JPEG quantization tables.
\vspace{0.3cm}

\item  \textbf{Neighboring pixels relationship inconsistencies}: when an image is spliced some artifacts can be created.
\vspace{0.3cm}
\item \textbf{Intrinsic image properties inconsistencies}: e.g. scene lights, shadows or perspective.
\end{enumerate}

\end{tframe}

%-- Light based detection --%

\begin{tframe}{Lighting-based inconsistencies}
\vspace{0.2cm}
Methods based on \textbf{Lighting inconsistencies} are particularly \emph{robus}: a perfect illumination adjustment in a image composition is very hard to achieve.
\vspace{0.3cm}
\begin{center}
\includegraphics[width=0.5\textwidth]{images/ducks.jpg}
\vspace{0.1cm}
\includegraphics[width=0.5\textwidth]{images/cakes.jpg}
\end{center}
\end{tframe}

\begin{tframe}{Lighting-based inconsistencies}
\vspace{0.1cm}
These methods can be divided into two types of approaches:
\vspace{0.2cm}
\begin{enumerate}
\item \textbf{Object light inconsistencies}: {\small involving light source position estimation or a full illumination model reconstruction. It can be done using \emph{shadows}, \emph{face geometry}, \emph{generic object surfaces}}.
\vspace{0.2cm}

\item \textbf{Illuminant colors inconsistencies}: {\small assuming that a scene is lit by the same light source, all objects must have the same illuminant colors.}
\vspace{0.3cm}
\end{enumerate}
\begin{center}
\includegraphics[width=0.5\textwidth]{images/lighting-based.jpg}
\end{center}
\end{tframe}


