%%%  Pacchetto di stile A4  %%%
\documentclass[12pt,a4paper,oneside]{book}
\usepackage[T1]{fontenc} 
\usepackage[utf8]{inputenc}
\usepackage{phdthesis} % Pacchetto di stile per le tesi

% Ridefiniamo la riga di testa delle pagine:
\usepackage{fancyhdr}
\pagestyle{fancy}
\usepackage{titlesec}
\usepackage{setspace}
\usepackage{wrapfig}
\usepackage{sectsty}
\usepackage{adjustbox}
\usepackage{lipsum}
\usepackage{multicol}
\usepackage[font=footnotesize]{caption}
\usepackage{listings} %Per inserire codice
\usepackage[usenames]{color} %Per permettere la colorazione dei caratteri 
\definecolor{dkgreen}{rgb}{0,0.6,0}
\definecolor{gray}{rgb}{0.5,0.5,0.5}
\definecolor{mauve}{rgb}{0.58,0,0.82}

\definecolor{lightgray}{rgb}{.9,.9,.9}
\definecolor{darkgray}{rgb}{.4,.4,.4}
\definecolor{purple}{rgb}{0.65, 0.12, 0.82}
\renewcommand\lstlistingname{}
\lstdefinelanguage{javascript}{
  keywords={typeof, new, true, false, catch, function, return, null, catch, switch, var, if, in, while, do, else, case, break, for},
  keywordstyle=\color{blue}\bfseries,
  ndkeywords={class, export, boolean, throw, implements, import, this},
  ndkeywordstyle=\color{darkgray}\bfseries,
  identifierstyle=\color{black},
  sensitive=false,
  comment=[l]{//},
  morecomment=[s]{/*}{*/},
  commentstyle=\color{purple}\ttfamily,
  stringstyle=\color{red}\ttfamily,
  morestring=[b]',
  morestring=[b]"
}

\lstset{frame=tb,
  captionpos=b,
  abovecaptionskip=0.5cm,
  aboveskip=0.8cm,
  belowskip=0.3cm,
  showstringspaces=false,
  columns=flexible,
  basicstyle={\small\ttfamily},
  numbers=left,
  numberstyle=\tiny\color{gray},
  keywordstyle=\color{blue},
  commentstyle=\color{dkgreen},
  stringstyle=\color{red},
  breaklines=true,
  breakatwhitespace=true
  tabsize=4,
  frame=lrbt,
  rulecolor=\color{gray}
}
\renewcommand{\chaptermark}[1]{\markboth{#1}{}}
\renewcommand{\sectionmark}[1]{\markright{\thesection\ #1}}
\fancyhf{}
\fancyhead[RO]{\thepage}
\fancyhead[LO]{\leftmark}
\renewcommand{\headrulewidth}{0.1pt}
\renewcommand{\footrulewidth}{0pt}
\headsep=20pt
\sectionfont{\Large} 
\subsectionfont{\itshape\large} 
\titleformat{\chapter}[display]
  {\normalfont\huge\bfseries}{\chaptertitlename\ \thechapter}{20pt}{\huge}
\titlespacing*{\chapter}{0pt}{-10pt}{40pt}

\setlength{\skip\footins}{1cm}
\setlength{\footnotesep}{0.4cm}

\graphicspath{{./figure/}}	% path directory figure
\linespread{1.6}	% interlinea
%%%%%%

\usepackage{graphics,graphicx}

\usepackage{amsmath,amsfonts,amssymb,amsthm} % pacchetti tipici per scrivere matematica  
\usepackage{latexsym}
\usepackage{array}
\usepackage{tocbibind}
\usepackage{listings}  % listati di codice

\usepackage{cite}
\usepackage{algorithm}
\usepackage{algpseudocode}

\makeatletter
\def\BState{\State\hskip-\ALG@thistlm}
\makeatother

\usepackage{color}

\usepackage{textcomp}
\usepackage{subfigure}
\usepackage{subfloat}

% TOC INFO: sono le info salvate nelle proprieta' del file pdf
\hypersetup{ 
  pdftitle=Master Thesis in Computer Science,
  pdfauthor=Lorenzo Cioni,
  pdfsubject=Illuminant map analysis for image splicing detection
}    

%%%  Frontespizio  %%%

\university{}	% Universita' degli Studi di xxx (Firenze)
\faculty{Scuola di Ingegneria} %Nome della Scuola
\degree{}
\dept{} % nome dipartimento
\course{Corso di Laurea Magistrale in Ingegneria Informatica} %Corso di Laurea

\accademicyear{Anno Accademico 2015/2016} %anno accademico
\supervisor{Prof. Alessandro Piva} % relatore
\supervisor{Prof. Carlo Colombo} %correlatore
\advisor{Dott. Massimo Iuliani} %correlatore
\advisor{Dott. Marco Fanfani} %correlatore
\author{Lorenzo Cioni}

%titolo in italiano
%\title{\textbf{{\Huge Analisi della sorgente luminosa per la rivelazione di fotomontaggi}}}

%titolo in inglese
%\englishtitle{\textbf{{\Huge Illuminant map analysis for image splicing detection}}}
\title{\textbf{{\Huge Illuminant map analysis for image splicing detection}}}

\setcounter{tocdepth}{3}
\setcounter{secnumdepth}{3}

%%%  BEGIN DOCUMENT  %%%
\begin{document}

%%%  FRONT MATTER  %%%
\frontmatter
\maketitle  	% stampa la pagina di frontespizio

\chapter*{Abstract}
\addcontentsline{toc}{chapter}{Abstract}
\chaptermark{Abstract}

With the advent of powerful image editing tools, manipulating images and changing their content is becoming a trivial task. Unfortunately, most of the times, these modifications are aimed deceiving viewers, changing opinions or even affecting how people perceive reality. Therefore, the development of efficient and effective forgery detection techniques has become a hot research topic. 

From all types of image forgeries, image compositions are specially interesting. This type of forgery uses parts of two or more images to construct a \emph{new reality} from scenes that never happened. Among all different telltales investigated for detecting image compositions, illumination inconsistencies are considered the most promising since a perfect light matching in a forged image is still difficult to achieve. 

This thesis builds upon the hypothesis that image illumination inconsistencies are strong and powerful evidence of image composition and presents an extension of two approaches in literature, both based on the illuminant maps analysis. The first method is specific for human face forgeries, the second is a blind approach for detecting regional splicing.


\chapter*{Sommario}
\addcontentsline{toc}{chapter}{Sommario}
\chaptermark{Sommario}

Con l'avvento di sempre più potenti strumenti di modifica di immagini, la loro manipolazione è diventata un'operazione molto comune. Purtroppo, il più delle volte, queste modifiche sono volte ad ingannare gli spettatori, cambiare le loro opinioni o perfino a modificare il modo in cui percepiscono la realtà. Di conseguenza, lo sviluppo di tecniche sempre più efficienti ed efficaci per la rilevazione di fotomontaggi  è diventato un argomento di ricerca molto importante. 


Tra tutti i tipi di falsificazione, la composizione di più immagini è particolarmente interessante. Questo tipo di operazione utilizza le parti di due o più immagini al fine di costruire una vera e propria nuova realtà, ricostruire scene mai accadute realmente. Tra tutte le diverse tecniche di rilevazione di composizione di immagini, quelle basate sulle inconsistenze della luce sono considerate fra le più promettenti, poichè è quasi sempre difficile riuscire a rispettare la coerenza di illuminazione all'interno della stessa immagine quando questa viene modificata.

Questa tesi pone le sue basi proprio sull'ipotesi che l'individuazione di inconsistenze nella luce presenti in un'immagine sia una prova evidente di contraffazione. A partire da questa ipotesi, è presentata un'analisi ed un'estensione di due approcci presenti in letteratura che pongono le loro basi sulle \emph{Illuminant Maps}. Il primo metodo è specifico per la rilevazione di contraffazioni che riguardano volti umani nelle immagini; il secondo invece è un approccio molto più generico che cerca di individuare zone dell'immagine che presentano delle inconsistenze nel colore dell'illuminante.

% DEDICA
\newpage{\thispagestyle{empty}\null\vfil
\begin{flushright}
\textit{{\large Dedica}}
\end{flushright}
}

\tableofcontents

%% Corpo della tesi

\mainmatter{
\chapter*{Introduction}
\addcontentsline{toc}{chapter}{Introduction}
\chaptermark{Introduction}

Recently advanced image processing tools and computer graphics techniques make it straightforward to edit or modify digital images. In a forensics scenario, this raises the challenge of discriminating original images from malicious forgeries. Particular region from an image is pasted into other image with purpose to create image splicing. 

Image splicing is a common type of image tampering (manipulation) operation. The image integrity verification as well as identifying the areas of tampering on images without need to any expert support or manual process or prior knowledge original image contents is now days becoming the challenging research problem.

Investigating image's lighting is one of the most common approaches for splicing detection. This approach is particularly robust since it's really hard to preserve the consistency of the lighting environment while creating an image composite (i.e. a splicing forgery). 

In this scenario, there are mainly two main approaches:
\begin{enumerate}
\item based on the object-light geometric arrangement
\item based on illuminant colors
\end{enumerate}

We focused our attention on the illuminant-based approach, which assumes that a scene is lit by the same light source. More light sources are admitted but far enough such as to produce a constant brightness across the image. In this condition, pristine images will show a coherent illuminant representation; on the other hand, inconsistencies among illuminant maps will be exploited for splicing detection. 

\emph{Illuminant Maps} locally describes the lighting in a small region of the image. In the computer vision literature exists many different approaches for determining the illuminant of an image has been proposed. In particular, such techniques are divided into two main groups: statistical-based and physics-based approaches.

Regarding the first group, we start investigating on the \emph{Grey-World algorithm} \cite{Buchsbaum19801}, which is based on the Grey-World assumption, i.e. the average reflectance in a scene is achromatic. In \cite{finlayson2004shades}, this algorithm proved to be special instances of the Minkowski-norm. Van de Weijer et al. \cite{van2007edge} than proposed an extension of the Gray-World assumption, called \emph{Gray-Edge hypothesi}s \cite{van2007edge}, which assumes that the average of the reflectance differences in a scene is achromatic. The reflectance differences can be determined by taking derivatives of the image. Therefore, the authors present a framework with which many different algorithms can be constructed.
We focus our attention on the last case, called generalized \emph{Grey-World algorithm (GGE)}. The resulting illuminant maps presents also global illuminant features because of the gray-world and grey-edge assumptions.

For the latter group, was investigate the method proposed by Riess et al. \cite{riess2010scene}, which extends the \emph{Inverse Intense Chromaticity} (\emph{IIC}) space approach proposed by Tan et al. \cite{tan2004color} and tries to model the illuminants considering the dichromatic reflection model \cite{tominaga1989standard}. In this case, the illuminant map is evaluated dividing  images into blocks, named superpixels, of approximately the same object color, then the illuminant color is evaluated for each block solving the lighting models locally. 

Carvalho et al. \cite{carvalho2016illuminant} then presents a method that relies on a combination of the two approaches for the detection of manipulations on images containing human faces. In addition to maps, a large set of shape and texture descriptors are used together. Note that, from a theoretical viewpoint, it is advantageous to consider only image regions that consist of approximately the same underlying material: for this reason, in \cite{carvalho2016illuminant} the authors focused their analysis on human faces.

In \cite{carvalho2016illuminant} it is also shown that the difference between the two maps, GGE and IIC, increased when fake images are processed. This insight leads to the idea that it is possible to localize tampered image regions simply by considering IM differences with some metric, avoiding the computation of multiple descriptors.
\chapter{Related work}

\section{Image forgery}

When dealing with a digital image, it is quite common to wonder if it is original or has been counterfeited in some way. Images and videos have become the main information carriers in the digital era and used to store real world events, but  they are very easy to manipulate because of the availability of the powerful editing software and sophisticated digital cameras.

The contexts where doctored pictures could be involved are very disparate; they could be used in a tabloid or in an advertising poster or included in a journalistic report but also in a court of law where digital (sometimes printed) images are presented as crucial evidences for a trial in order to influence the final judgement. So, especially in the last case, reliably assessing image integrity becomes of fundamental importance. 

\emph{Image forensics} specifically deals with such issues by studying and developing technological tools which generally permit determining, by only analyzing a digital photograph (i.e., its pixels), if that asset has been manipulated or even which could have been the adopted acquisition device (such an issue is not relevant to the topic of the present paper). Moreover, if it has been established that something has been altered, it could be important to understand in which part of the image itself such a modification occurred, for instance, if a person or a specific object has been covered, if an area of the image has been cloned, if something (i.e., a face or a weapon) has been copied from another different image, or, even more, if a mixture of these processes has been carried out. 

\section{Image forgery detection techniques}

To verify the authenticity of a picture many techniques have been identified that can be categorized into active (intrusive) and blind (non-intrusive).

Operative techniques involve a phase of preprocessing the image itself at the time of its creation in order to include some additional information that will be used during the analysis phase. An example of operative technique is the watermarking.

Passive techniques analyze the content of the image using various statistics or semantic content in order to identify inconsistencies of some kind. This approach does not alter the contents of the image.
There is a general technique, suitable to capture all kinds of inconsistencies present in an image, but each different method specialises in the identification of a particular type.

\section{Image splicing}

Image splicing is a very common type of infringement which basically consists in copying a region of a given image to another, thus creating a composition of two different pictures together.

\begin{figure}
  \centering
    \includegraphics[width=0.8\textwidth]{imagesplicing}
    \caption{Image splicing process}
\end{figure}


\subsubsection{Some famous cases}

Photography has lost its innocence since the early days of his birth. In fact already in 1860, only a few decades after Niépce created the first photo, the first manipulated photographs were identified in 1826. With the advent of digital cameras, camcorders and sophisticated photo editing software, digital image manipulation is becoming more common. 

\paragraph{O.J. Simpson - June 1994}

This altered photography O.J. Simpson appeared on the cover of the magazine Time Magazine, soon after his arrest for murder. 

\begin{wrapfigure}{r}{0.5\textwidth}
  \begin{center}
    \includegraphics[width=0.48\textwidth]{ojsimpson}
  \end{center}
  \caption{The Time Magazine and O.J. Simpson}
  \vspace{-1cm}
\end{wrapfigure}

In fact, the photograph was altered compared to the original image that has appeared on the cover of Newsweek magazine. Time magazine was accused of manipulation of the photography in order to make darker and menacing figure of Simpson.

\paragraph{Iraq - April 2003}

This composition of a British soldier in Basra, which keeps pointing toward a civilian Iraqi gesticulates covered, she appeared on the cover of the Los Angeles Times, immediately after the invasion of Iraq. 

\begin{wrapfigure}{l}{0.5\textwidth}
  \begin{center}
    \includegraphics[width=0.48\textwidth]{iraq}
  \end{center}
  \caption{An example of image composition}
\end{wrapfigure}

Brian Walski, a staff photographer for the Los Angeles Times and a veteran of the news with thirty years of experience, was summarily fired from his publisher for their merged two of his shots in order to improve the composition.

\paragraph{George W. Bush - March 2004}

This image, taken from promo released for the election campaign of George w. Bush, outlined a packed audience of soldiers as a backdrop to a child who was flying the American flag. This image was digitally souped-up, using a crude copy and paste, removing Bush from the podium. 

\begin{wrapfigure}{r}{0.5\textwidth}
  \begin{center}
    \includegraphics[width=0.48\textwidth]{bush}
  \end{center}
  \caption{An example of image composition}
\end{wrapfigure}


After admitting the tampering with the staff of the television station edited and sent to Bush promo with the original photo.

\section{Methods based on light inconsistencies}

\subsection{Illuminant Maps}

\subsubsection{Generalized Greyworld algorithm}

\subsubsection{Inverse Intensitiy Chromaticiy}


\subsection{Human faces splicing detection}

\subsection{Region splicing detection}
\chapter{Proposed approach}

In the previous chapter, a review of two splicing detection methods based on illuminant colors analysis has been presented. However, their effectiveness still needed to be improved for real forensic applications.

The approach proposed in this chapter has been developed to correct some drawbacks and mainly to achieve an improved accuracy over the two approaches presented in Chapter 1. 

\section{Overview}

Most of the times, the splicing detection process relies on the expert's experience and background knowledge. This process usually is time consuming and error prone once that image splicing is more and more sophisticated and an aural (e.g., visual) analysis may not be enough to detect forgeries.

This approach to detecting image splicing is developed aiming at minimizing the user interaction. 

The two methods, presented in the previous chapter \cite{carvalho2016illuminant} and \cite{fan2015image}, are now being used in synergy with each other, going to analyze each image at the same time looking for potential signs of forgery.

Starting from an image we want to analyze, the method will output a set of results.
\begin{itemize}
\item A classification \textbf{label} indicating whether an image is believed to be original or counterfeit.
\item A classification \textbf{score} indicating the confidence of the method output.
\item A \textbf{detection map} highlighting the detected spliced regions.
\end{itemize}

The proposed approach minimizes human interaction being fully automated. However, not both the modules can operate in any circumstance. The face splicing detection module will work only if there is a number of faces greater than or equal to two.

\section{Face splicing detection module}

The first module is to implement, with a few enhancements and simplifications, the method proposed by Carvalho et al. \cite{carvalho2016illuminant} and presented in Section 1.6.

The algorithm to analyze an image that contains at least two sides, and whether the faces have changed or not. In particular, in the case of a fake, you can tell which of the faces in the image is false.

The method requires an initial training phase, in which you must have a dataset with faces noted. The considered dataset is the DSO-1.

This module consists of 4 consecutive stages:

\begin{enumerate}
\item \textbf{Illuminant maps extraction}: given an input image, two different illuminant maps are evaluated, using the IIC and GGE extraction methods.
\item \textbf{Face detection}: after the illuminant maps extraction, the human faces in the image are detected. In the training phase the faces positions are read from the groundtruth file. If the given image contains less than two faces, it is discarded.
\item \textbf{Paired face feature extraction}: human faces are considered and classified in pairs. From each extracted face in the previous step, a color descriptor is used in order to extract features.
\item \textbf{KNN models training}: fixed a value of $K$, a set of KNN models are trained using previous feature vectors.
\item \textbf{Forgery detection and classification}: in this step an image is classified as fake or normal. Given an image classified as fake, the analysis is refined pointing out which part of the image is the result of a composition.
\end{enumerate}

\subsection{Illuminant maps extraction}

Given a single image $I$ the two different illuminant maps are extracted: both generalized grayworld algorithm (GGE) and inverse-intensity chromaticity estimation (IIC) are used.

\begin{figure}[!htb]
\minipage{0.32\textwidth}
  \includegraphics[width=\linewidth]{splicing-33.jpg}
  \caption{The original image}\label{fig:awesome_image1}
\endminipage\hfill
\minipage{0.32\textwidth}
  \includegraphics[width=\linewidth]{splicing-33_gge_map.jpg}
  \caption{GGE illuminant map}\label{fig:awesome_image2}
\endminipage\hfill
\minipage{0.32\textwidth}%
  \includegraphics[width=\linewidth]{splicing-33_iic_map.jpg}
  \caption{IIC illuminant map}\label{fig:awesome_image3}
\endminipage
\end{figure}

Differently from de Carvalho et al. \cite{carvalho2016illuminant}, who have used the conversion of IMs to the YCbCr color space, the two IMs are considered as two different color maps theirself.

\subsection{Face detection}

One of the major drawback of the method proposed by Carvalho et al. \cite{carvalho2016illuminant} relies on the user interaction needed for detecting human faces in the image. In this approach a common face detection module is used to achieve the same results, aiming at minimizing the user dependency and making the entire algorithm user independent.

The face detector used is the one initially proposed by Viola and Jones \cite{viola2001rapid}\cite{viola2004robust} and improved by  Lienhart et al. \cite{lienhart2002extended}. Usually called simply Viola-Jones, its original motivation was face detection, but it can be trained to detect different object classes. 

This detector combines four key concepts: 
\begin{itemize}
\item Simple rectangular features, called \emph{Haar features}
\item \emph{Integral Image}\cite{crow1984summed} concept for rapid feature detection 
\item \emph{AdaBoost}\cite{freund1995desicion} machine-learning method 
\item A \emph{cascade classifier} to combine all features efficiently 
\end{itemize}

The used rectangle combinations are not true \emph{Haar wavelets}\cite{haar1910theorie}. Instead, they contain better suited rectangle combinations used for visual object detection. The presence of a Haar feature is determined by subtracting the pixel values of the dark region to the pixel values of the light one. If the difference exceeds some threshold value set during the training process, the feature is said to be present. 

\begin{figure}[h!]
  \centering
    \includegraphics[width=0.5\textwidth]{facedetected}
    \caption{The output of the face detection module}
    \label{fig:facesdetected}
\end{figure}

\emph{OpenCV}\footnote{OpenCV - Open source computer vision. http://opencv.org} provides an implementation of the Viola-Jones face detector as \emph{cvHaarDetectObjects}.

Figure \ref{fig:facesdetected} shows the output of the face detector module. Given the faces bounding boxes, the corresponding regions in the illuminant maps are extracted.

\subsection{Paired face feature extraction}

From each extracted face in the previous step, we need to find telltales that allow identification of spliced images. 

According to Riess and Angelopoulou \cite{riess2010scene}, when the output illumination maps are analyzed by an expert for detecting forgeries, the main observed feature is color. Thus, differently from Carvalho et al. \cite{carvalho2016illuminant} who have explored a large set of descriptors of different visual properties (e.g., texture, shape, color, among others) we focused our attention on color descriptors.

The considered color description techniques are ACC \cite{huang1997image}, BIC \cite{stehling2002compact}, CCV \cite{pass1997comparing}, and LCH \cite{swain1991color}.

\subsubsection{ACC descriptor}

\emph{Color Autocorrelogram (ACC) }\cite{huang1997image} maps the spatial information of colors by pixels correlations in different distances. Let $I$ an image, the \emph{autocorrelogram} (the term \emph{“correlogram”} is adapted from spatial data analysis \cite{upton1985spatial}) computes the probability of finding two pixels in $I$ with a color $c$ in a distance $d$ from each other. 

After the autocorrelogram computation, a set of $m$  probability values for each distance $d$ are considered, where $m$ stands for the number of colors in the quantized space. 

The implemented version quantized the RGB color space into 64 bins and considered 4 distance values (1, 3, 5, and 7). The \emph{L1} distance function is used.

\subsubsection{BIC descriptor}

\emph{Border/Interior Pixel Classification (BIC)} \cite{stehling2002compact} is a region-based color descriptor. Its extraction algorithm classifies the image pixels in border or interior
pixels. 

The image is first quantized into 64 colors in RGB color space. Then, each pixel is classified as interior
if its neighbors (above, below, left and right pixels) have the same color. Otherwise it is classified as border pixel. After the classification, two histograms are generated: one for border pixels and other for interior pixels. These histograms are stored as one single histogram with 128 bins. The distance
function is called \emph{dLog} and it compares histograms in a logarithmic scale. 

\subsubsection{CCV descriptor}

\emph{Color Coherence Vector (CCV)} \cite{pass1997comparing} is a very popular color descriptor in the literature. Its extraction algorithm classifies the image pixels in coherent or incoherent pixels. This classification considers if the pixel belongs or not to a region with similar colors, called \emph{coherent region}. After the classification, two color histograms are computed: one for coherent pixels and other for incoherent pixels.

Both histograms are concatenated to compose the final feature vector. The RGB color space is quantized into 64 bins and L1 distance function is used.

\subsubsection{LCH descriptor}

\emph{Local Color Histogram (LCH)} \cite{swain1991color} is one of the most popular descriptors that is based on fixed size regions to describe image properties. Its extraction algorithm splits the image into fixed size blocks and computes a color histogram for each region. After that, the histograms of each region
are concatenated to compose one single histogram. 

The implemented version splits the image
in 16 regions (4x4 grid) and quantized the RGB color space into 64 bins. This generated feature vectors with 1024 values. The L1 distance function is used.


\subsubsection{Paired features}

In order to detect a face forgery, we need to compare  the current analyzed image region against other, looking for color inconsistencies. Thus, instead of considering each face separately, a paired face feature vector is encoded for every possible face pair.

Given two face descriptors, $\mathcal{D_1}$ and $\mathcal{D_2}$, whose length depends on the color descriptor used, the final descriptor is computed concatenating them into a single feature vector $\mathcal{P}$. So, in an image $I$ containing $q$ faces, a set $S$ of feature vectors are extracted, with

$$
S = \{\mathcal{P}_1, \ldots, \mathcal{P}_m\} \quad \textrm{  where } m = \frac{q (q-1)}{2}
$$

if $q \geq 2$. 

\subsection{KNN models training}

\subsection{Forgery detection and classification}


\section{Region splicing detection module}

The second form of the algorithm is to implement the method proposed by Fan et al.\cite{fan2015image} with some changes in order to try to correct some of its major drawbacks presented in Section 1.7.

The splicing detection task performed by our approach consists in labelling a new image among two pre-defined classes (real and fake) and later pointing the face with higher probability to be the fake face. In this process, a classification model is created to indicate the class to which a new image belongs.

In summary, this module consists of the 6 main steps:

\begin{itemize}
\item \textbf{Image segementation}: relies on vertical and horizontal image segmentations. The outputs of this stage are two set of directional image bands. 
\item \textbf{Band illuminant estimation}: consists in estimating the illuminant color for each segmented band using 5 different GGE algorithms.
\item \textbf{Reference illuminant estimation}: consists in estimating the illuminant reference value for each direction.
\item \textbf{Feature vector evaluation}: relies on encoding the singular band illuminant information into a feature vector for further classification. The feature vector elements are the differences between the current illuminant color and the reference one.
\item \textbf{Band classification}: consists in labelling each image band into one of the know classes (real or fake) based on the previously learned classification model.
\item \textbf{Detection map}: using the classification output of the previous step, a detection map is build. The higher the value of this map, the higher the  resulting classification score for a single pixel.
\end{itemize}

\begin{figure}[h!]
  \centering
    \includegraphics[width=1\textwidth]{pipeline_regions}
    \caption{Image regional splicing detection module pipeline}
    \label{fig:regionsmodulepipeline}
\end{figure}

The Fig \ref{fig:regionsmodulepipeline} summarize the module pipeline.

\subsection{Image segmentation}

In the first step of the process, the input image is segmented in order to obtain two image bands categories: horizontal and vertical bands. This kind of segmentation method is chosen because of its simplicity.

First, a band width $B_w$ and a band height $B_h$ is set. Due to the fact that the segmentation has to produce overlapping bands, we set a delta factor of $\frac{1}{4}$. As a result we get overlapping stripes for a total of 25\% of their area. 

The choice of the band size is a crucial phase of the algorithm: a band too tight would fail to capture the information necessary to classify our object of interest as falsified, unlike a band too wide would capture instead too much additional information.

The choice of the overlapped area percentage makes possible a more detailed evaluation. In this way it is possible to classify the same region of the image more than once, increasing the expressive power of our final classifier, the detection map.
 
In summary, let $I$ be the input image. After the segmentation process we obtain a set $B$ of bands containing all vertical and horizontal bands:

$$
B = \{V_1, \ldots, V_n, H_1, \ldots, H_m\}
$$

The dimension of $B$ is given by the sum of the number of vertical ($n$) and horizontal ($m$) bands.

\subsection{Band illuminant estimation}

The resulting image bands are now processed in order to evaluate the illuminant color using different techniques.

For this step, the Generalized Grayworld \cite{van2007edge} algorithms are used, as presented in Chapter 1. For each band, the illumination estimation is accomplished by using one of algorithms composed of Grey-World, Max-RGB, Shades of Grey, first-order Grey-Edge and second-order Grey-Edge. Thus we have 5 illuminant estimates for each band.
Table \ref{table:ggemethods} in Chapter 1 summarize this algorithm parameters.

In this way we obtain 5 different illuminant estimation for each single band.

$$
\forall b \in B \qquad R_a(b) = GGE_a(b) \qquad a \in \mathcal{A}
$$

where $\mathcal{A}$ is the set of the previously mentioned algorithms and $GGE$ is the algorithm implementation.

\subsection{Reference illuminant estimation}

After estimating the illuminant of each horizontal/vertical band, it is evaluated a reference illuminant color for each used algorithms.

$$
\forall a \in \mathcal{A} \qquad RV_a = median(R_a(b)) \quad \forall \; b \textrm{ vertical}
$$
$$
\forall a \in \mathcal{A} \qquad RH_a = median(R_a(b)) \quad \forall \; b \textrm{ horizontal}
$$

where $RV_a$ and $RH_a$ are the reference illuminant colors for the $a$ algorithm for vertical and horizontal direction respectively. As these two values are calculated using the median, will be identical to one of the values of a band of that same direction.

\subsection{Feature vector estimation}

Given the two reference color for each of the two directions, a feature band for each single image band can be built.

Assuming a single light source in the image, all the evaluated illuminant will point at the same color.
Based on this assumption, a feature vector that capture how the gang present of singularities in the illuminant is built.

Let $b \in B$ a single band lying on $d \in \{vertical, horizontal\}$ direction. The feature vector for $b$ will be:

\begin{equation}\label{eq:regionsfeaturevector}
f_{b} = \{m_1, m_2, m_3, m_4, m_5\}
\end{equation}
where
$$
m_i = dist(R_a(b) - RC)
$$
where $RC = RV_a$ or $RC = RH_a$ in case of vertical and horizontal band respectively and $dist$ is the Euclidean distance function between two RGB values.

\subsection{Band classification}

Given a feature vector, a machine learning approach is used to automatically classify the band. 
In a prior step, a classification model is trained using a set of sample data. A Support Vector Machine (SVM) classifier with a radial basis function (RBF) kernel is used.

$$
\forall \; b \in B \quad Label(b) = SVM(b)
$$

\subsection{Detection map}

In order to collect all the classification outputs, a detection map is build and updated after each evaluation. If the result of the classification is \emph{positive} (\emph{fake}), all the pixel values that belong to the band portion in the image will be increased by one unit.

At the end of the process, all the splicing region pixels will have greater values than the others. At this stage a color map is displayed to give a visual feedback for locating the splicing image parts.



\chapter{Experiments and results}

This chapter describes the experiments we performed to show the effectiveness of the proposed method. 

\section{Evaluation datasets}
To quantitatively evaluate the proposed methods, and to compare it to related work,
we considered two datasets: the DSO-1 and DSI-1 datasets\footnote{Public available for download at https://recodbr.wordpress.com/code-n-data}. Each dataset comes with a face position groundtruth and a splicing region mask. Additionally, the ColorChecker dataset \cite{gehler2008bayesian} is used as a source of pristine images.

The latter dataset has been mainly used for experimenting on pristine data: due its characteristics it is very varied and lends itself well to image analysis based on color.

\subsection{DSO-1}

The DSO-1 dataset is composed of 200 indoor and outdoor images with the resolution of 2048 × 1536 pixels. Out of this set of images, 100 are original, i. e. they have no adjustments, and 100 are forged. 

\begin{figure}[!htb]
\minipage{0.48\textwidth}
  \includegraphics[width=\linewidth]{dso_sample}
  \caption{DSO-1 sample original image}\label{fig:dsooriginalimage}
\endminipage\hfill
\minipage{0.48\textwidth}
  \includegraphics[width=\linewidth]{dso_sample_spliced}
  \caption{DSO-1 sample spliced image}\label{fig:dsosplicedimage}
\endminipage
\end{figure}

The forgeries were created by adding one or more individuals in a source image that already contained one or more people. When necessary, some post-processing operations were made (such as color and brightness adjustments) in order to increase photorealism.

\subsection{DSI-1}

The DSI-1 dataset is composed of 50 images with different resolutions (25 original and 25 doctored) downloaded from different websites in the Internet. Original images were downloaded from Flickr and doctored images were collected from different websites such as \emph{Worth 1000}, \emph{Benetton Group 2011}, \emph{Planet Hiltron}, etc.

\begin{figure}[!htb]
\minipage{0.46\textwidth}
  \includegraphics[width=\linewidth]{dsi_sample_normal}
  \caption{DSI-1 sample original image}\label{fig:dsioriginalimage}
\endminipage\hfill
\minipage{0.50\textwidth}
  \includegraphics[width=\linewidth]{dsi_sample_spliced}
  \caption{DSI-1 sample spliced image}\label{fig:dsisplicedimage}
\endminipage
\end{figure}

\subsection{ColorChecker}

The ColorChecker dataset is a collection of images for evaluating Color Constancy algorithms built as additional material to \cite{gehler2008bayesian}. It consists in 568 RGB colored images of different scenes, both indoor and outdoor taken under different illuminations. In each scene a \emph{Gretag MacBeth Color Checker Chart}\footnote{The \emph{ColorChecker} consists of a series of six gray patches, plus typical additive (Red-Green-Blue) and subtractive (Cyan-Magenta-Yellow) primaries, plus other \emph{natural} colors such as light and dark skin, sky-blue, foliage, etc. It is usually used for color calibration of digital cameras. The color pigments were selected for optimum color constancy when comparing pictures of the chart with pictures of the natural colors. The ColorChecker was introduced in a 1976 paper by McCamy, Marcus, and Davidson in the Journal of Applied Photographic Engineering \cite{mccamy1976color}.}  was placed such that it was illuminated by the main scene illuminant and thus its color could be retrieved. The data is available in Canon RAW format free of any correction.

\begin{figure}[h!]
  \centering
    \includegraphics[width=0.65\textwidth]{colorchecker_sample}
    \caption{Example of an image of the ColorChecker dataset}
    \label{fig:colorcheckersample}
\end{figure}

\section{Face forgery detection module performance}

This section describes the experiments performed to evaluate the face forgery detection module. In these experiments, the DSO-1 and the DSI-1 datasets are used for evaluation.

After characterizing an image with a specific image descriptor, the next step consists of using an appropriate learning method.

the proposed method method focuses on using complementary information, the color descriptors, to describe the Illuminant Maps. Accordingly to Carvalho \emph{et al. }\cite{carvalho2016illuminant}, we selected the k-Nearest Neighbor (kNN) classifier instead of more powerful and computational intensive ones such as Support Vector Machines (SVM) due fact that, SVM and kNN, or even fusion of both learning methods, present very similar results, enforcing the choice for a simpler learning technique (kNN)\cite{carvalho2016illuminant}. The value of \emph{k} is set to 5. 

For all image descriptors, we have used the standard configuration proposed by Penatti \emph{et al.} \cite{penatti2012comparative}.

As described in Chapter 2, eight different kNN models are trained and used for the classification, considering all the possible combinations of the couples composed by a IMs transformed space (GGE e IIC)and a color descriptor (chosen from ACC, BIC, CCV and LCH).

\subsection{Color descriptors accuray}

Since the final classification output is given by majority voting of all the selected classifiers, the first round of experiments aimed at evaluating the accuracy of each image descriptors.

\begin{table}[h!]
\centering
\begin{tabular}{l c c c c c c} 
\hline \hline 
\textbf{Test case} & \textbf{Train} & \textbf{Test} & \textbf{ACC} & \textbf{BIC} & \textbf{CCV} & \textbf{LCH} \\ [0.5ex]
\hline
Test 1 & DSO-1 & DSO-1 &	0.75 & 0.75	& 0.72 & \textbf{0.78}\\
Test 2 & DSI-1 & DSI-1 &	0.78 & 0.79 & 0.77 & \textbf{0.82}\\
Test 3 &	DSO-1 &	DSI-1 &	0.56 & \textbf{0.57} & 0.53 & 0.53\\
Test 4 &	DSI-1 & DSO-1 & 0.58 & 0.55 & 0.51 & \textbf{0.59}\\ [1ex]
\hline
\end{tabular}
\caption{Accuracy for kNN technique using a single color descriptor. Experiments are performed using 10-fold cross-validation in test case number 1 and 2.}
\label{table:colordescriptorperformance}
\end{table}

Table \ref{table:colordescriptorperformance} shows the results of all tested combinations of train and test set. The classification results show that, generally, LCH color descriptor yielded the higher accurary, but there is not a descriptor that outperforms the others.

\subsection{Test cases}

After analyzing each single descriptor accuracy, we proceeded to evaluate the overall module using a combination of all of them, as described in Chapter 2. Since not all the color descriptors perform equally, a different weight is given at each classifier.

In the following experiments, the  method has been applied for classifying a face pair as fake or real using uniform (Table \ref{table:performancefacedet}) and non-uniform (Table \ref{table:performancefacedetnonun}) weights. Essentially, these experiments evaluate the forgery detection performances.

For each test case in the suite it is reported:
\begin{enumerate}
\item The training dataset (\textbf{\emph{Train}});
\item The evaluation dataset (\textbf{\emph{Test}});
\item The classification accuracy score (\textbf{\emph{Accuracy}});
\item The area under the R\emph{eceiver Operator Characteristic (ROC)} curve (\textbf{\emph{AUC}})
\item The accuracy score expressed through the $F_1$ score (also known as \textbf{\emph{F-Score}})
\end{enumerate}

Accordingly with Jeni \emph{et al.} \cite{jeni2013facing}, when dealing with AUC score to measure the performance of classifier, one of the major drawbacks relies on the fact that an increasing of AUC doesn't really mean a better classifier, but it could be just the side-effect of too many negative examples used in training.

Therefore, another classification accuracy score is also provided, the \emph{F-Score}. Given $TP$, $FP$ and $FN$ the true positives, false positives and false negatives values. 

Let the classification \emph{precision} score given by

$$
precision = \frac{TP}{TP + FP}
$$

where $TP$ and $FP$ are true positives and false positives respectively.

The \emph{recall} value is defined as
$$
recall = \frac{TP}{TP + FN}
$$
where $FN$ stands for false negatives. The $F_{\beta}$ score is defined as the harmonic mean of precision and recall values:

\begin{equation}
F_{\beta} = (1 + \beta^2) * \frac{precision * recall}{(\beta^2 * precision) + recall}
\end{equation}

where $\beta$ states for the relative importance given to precision comparing to recall. In our experiments, we considered $\beta = 1$ (i.e. the $F_1$ score), so:
\begin{equation}
F_{1} = 2 * \frac{precision * recall}{precision + recall}  = \cdots = \frac{2 * TP}{2 * TP + FP + FN}
\end{equation}

In order to proceed with the comparison between using uniform and non-uniform weights for classifiers, we have to determine the most appropriate weight values for each descriptor. The weights have been found with an exhaustive search by evaluating the performance of the currently analyzed weights combination over the DSO-1 dataset with a 10-fold cross-validation protocol. 

To reduce the computational cost, we limited the range of values to be explored for each classifier basing on the results reached for each single descriptor, presented in Table \ref{table:colordescriptorperformance}, giving more importance to LCH descriptor.

Experimental results are collected in Table \ref{table:performancefacedet}, for the uniform weights case, and in Table \ref{table:performancefacedetnonun}, for non-uniform case. 

\begin{table}[h!]
\centering
\begin{tabular}{l c c c c c} 
\hline \hline 
\textbf{Test case} & \textbf{Train} & \textbf{Test} & \textbf{ACC} & \textbf{AUC} &\textbf{ F-Score} \\ [0.5ex]
\hline
Test 1 & DSO-1 & DSO-1 &	0.82 & 0.88	& 0.77\\
Test 2 & DSI-1 & DSI-1 &	0.87 & 0.92 & 0.87\\
Test 3 &	DSO-1 &	DSI-1 &	0.58 & 0.58 & 0.62\\
Test 4 &	DSI-1 & DSO-1 & 0.62 & 0.59 & 0.53\\ [1ex]
\hline
\end{tabular}
\caption{Performance of face forgery detection module over paired faces using uniform weights.}
\label{table:performancefacedet}
\end{table}

\begin{table}[h!]
\centering
\begin{tabular}{l c c c c c} 
\hline \hline 
\textbf{Test case} & \textbf{Train} & \textbf{Test} & \textbf{ACC} & \textbf{AUC} &\textbf{ F-Score} \\ [0.5ex]
\hline
Test 1 & DSO-1 & DSO-1 &	0.84 & 0.90	& 0.78\\
Test 2 & DSI-1 & DSI-1 &	0.89 & 0.92 & 0.89\\
Test 3 &	DSO-1 &	DSI-1 &	0.59 & 0.58 & 0.64\\
Test 4 &	DSI-1 & DSO-1 & 0.63 & 0.60 & 0.54\\ [1ex]
\hline
\end{tabular}
\caption{Performance of face forgery detection module over paired faces using non-uniform weights.}
\label{table:performancefacedetnonun}
\end{table}

Resulting accuracy scores are obtained as average over 5 consecutive runs of the algorithm. For a better visualization, Figure \ref{fig:compareknnweights} depicts a direct comparison between the accuracy of both results as a bar graph.

\begin{figure}[h!]
  \centering
    \includegraphics[width=0.9\textwidth]{compareknnweights}
    \caption{Classification accuracies comparison between using uniform and non uniform weights for kNN classifiers over the considered 4 test cases of Table \ref{table:performancefacedet}. The blue bins depict the reached accuracies using uniform weights, reds are for non-uniform case.}
    \label{fig:compareknnweights}
\end{figure}

These results show that, although we can notice a slight performance improvement in all test cases, the use of non-uniform weights do not affect much on the final classification results.

In the following subsections are described the experimental results for each analyzed test case using non-uniform weights. We show results using classical \emph{Receiver Operator Characteristic (ROC)} \cite{fawcett2006introduction} and Precision-Recall curves\cite{Davis:2006:RPR:1143844.1143874}. 

\subsubsection{Performance on DSO-1 dataset}

In this experiment, we applied the proposed method for classifying a face pair as fake or real considering only the DSO-1 dataset. We reached an average accuracy of 84\%.

\begin{figure}[!htb]
\minipage{0.42\textwidth}
  \includegraphics[width=\linewidth]{train_dso_crossvalidation}
\endminipage\hfill
\minipage{0.53\textwidth}
  \includegraphics[width=\linewidth]{train_dso_crossvalidation_prec_rec}
\endminipage
\caption{ROC curve and Precision-Recall curve for DSO-1 cross-validation classification of paired faces.}\label{fig:regiondetnormal}
\end{figure}

\subsubsection{Performance on DSI-1 dataset}

\begin{figure}[!htb]
\minipage{0.42\textwidth}
  \includegraphics[width=\linewidth]{train_dsi_crossvalidation}
\endminipage\hfill
\minipage{0.53\textwidth}
  \includegraphics[width=\linewidth]{train_dsi_crossvalidation_prec_rec}
\endminipage
\caption{ROC curve and Precision-Recall curve for DSI-1 cross-validation classification of paired faces.}\label{fig:regiondetnormal}
\end{figure}

\subsubsection{Cross dataset performance on DSO-1}

\begin{figure}[!htb]
\minipage{0.42\textwidth}
  \includegraphics[width=\linewidth]{train_dsi_test_dso}
\endminipage\hfill
\minipage{0.53\textwidth}
  \includegraphics[width=\linewidth]{train_dsi_test_dso_prec_rec}
\endminipage
\caption{ROC curve and Precision-Recall curve for DSO-1 paired faces classification using trained data on DSI-1.}\label{fig:regiondetnormal}
\end{figure}

\subsubsection{Cross dataset performance on DSI-1}

\begin{figure}[!htb]
\minipage{0.42\textwidth}
  \includegraphics[width=\linewidth]{train_dso_test_dsi}
\endminipage\hfill
\minipage{0.53\textwidth}
  \includegraphics[width=\linewidth]{train_dso_test_dsi_prec_rec}
\endminipage
\caption{ROC curve and Precision-Recall curve for DSI-1 paired faces classification using trained data on DSO-1.}\label{fig:regiondetnormal}
\end{figure}


\subsubsection{Forgery detection performance}

In the following experiments, we used the proposed module to detect the face with the highest probability of being the fake face in an image tagged as fake by the classifier. 

In this round of experiments, we assume that the input image $I$ has already been classified as fake by the classifier (i.e. at least one couple of faces has been classified as fake). 

We performed this kind of experiments using both DSO-1 and DSI-1 datasets. For each test in the suite it is reported:
\begin{itemize}
\item The training dataset (\textbf{Train})
\item The evaluation dataset (\textbf{Test})
\item The total number of faces detected in the images (\textbf{Faces})
\item The true positive rate (\textbf{TPR})
\item The classification accuracy (\textbf{ACC})
\item The classification precision (\textbf{PREC})
\item The classification recall (\textbf{REC})
\item The $F_1$ score (\textbf{F-Score})
\end{itemize}

Experimental results are collected in Table \ref{forgerydetections}.

\begin{table}[h!]
\centering
\begin{tabular}{l c c c c c c c} 
\hline \hline 
\textbf{N.} & \textbf{Train} & \textbf{Test} & \textbf{Faces} & \textbf{PREC} & \textbf{REC} & \textbf{ACC} & \textbf{F-Score} \\ [0.5ex]
\hline
1 & DSO-1 & DSO-1 &	540 & 0.58 & 0.89 & 0.81	& 0.64\\
2 & DSI-1 & DSI-1 &	133 & 0.56 & 0.95 & 0.75 & 0.66\\
3 &	DSO-1 & DSI-1 & 540 & 0.31 & 0.18 & 0.63 & 0.25\\ 
4 &	DSI-1 &	DSO-1 &	130 & 0.34 & 0.43 & 0.67 & 0.37\\[1ex]

\hline
\end{tabular}
\caption{Performance of face forgery detection module over single faces using non-uniform weights.}
\label{table:forgerydetections}
\end{table}


\section{Regions forgery detection module performance}

This section describes the experiments performed to evaluate the regions forgery detection module.

\subsection{Creating the training set}

Since the method is based on the classification of entire bands of images, in order to train the SVM classifiers were created two different training set with specific requirements:
\begin{itemize}
\item Each image must have only one spliced band (either horizontal or vertical)
\item The spliced region of the image must consist only int the whole band
\item The spliced region position in the image must be chosen randomly
\end{itemize}

Each image in the dataset must be shipped with its spliced band position as groundtruth (for performance evaluation).

For the two new training set, the DSO-1 and ColorChecker datasets were used respectively as source. Each training set is splitted into two subset containing only vertical or horizontal spliced bands.

\begin{algorithm}[!h]
\begin{algorithmic}[1]
\State Let $d \in \{vertical, horizontal\}$ a direction
\State Let $S$ be the band size
\For {each image $i \in SourceDataset$}
\State Select another image $j$ randomly from the same set with $i \neq j$
\State Resize $j$ to fit $i$
\State Select a band $b$ of direction $d$ randomly from $j$
\State Put $b$ in $i$ at the same original position
\State Save $i$ as image
\EndFor
\end{algorithmic}\caption{Spliced dataset creation algorithm}\label{alg:spliceddatasetcreation}
\end{algorithm}

The pseudocode used for generating the spliced datasets is proposed in Algorithm \ref{alg:spliceddatasetcreation}. That procedure is repeated for both DSO-1 and ColorChecher as $SourceDataset$ and for each direction. We called \emph{SplicedDSO} the dataset generated from DSO-1, \emph{SplicedCC} the one from ColorCherer. Both have 200 images vertically and horizontally spliced.

\begin{figure}[!htb]
\minipage{0.48\textwidth}
  \includegraphics[width=\linewidth]{splicedcchor}
\endminipage\hfill
\minipage{0.48\textwidth}
  \includegraphics[width=\linewidth]{splicedccver}
\endminipage
\caption{Example images from the generated \emph{SplicedCC} dataset. Left with horizontal spliced band, right with vertical.}\label{fig:splicedccsamples}
\end{figure}

\begin{figure}[!htb]
\minipage{0.48\textwidth}
  \includegraphics[width=\linewidth]{spliceddsohor}
\endminipage\hfill
\minipage{0.48\textwidth}
  \includegraphics[width=\linewidth]{spliceddsover}
\endminipage
\caption{Example images from the generated \emph{SplicedDSO} dataset. Left with horizontal spliced band, right with vertical.}\label{fig:spliceddsosamples}
\end{figure}

Figure \ref{splicedccsamples} and \ref{fig:spliceddsosamples} show some example of generated images from both dataset. 

\subsection{Evaluating module performance}

The proposed method output consists in a\emph{ detection map} related to the analyzed image. In order to evaluate the performance of the module, we consider the single pixels classification accuracies. The evaluation dataset for all the experiments is the original DSO-1 dataset (considering only the spliced images).

Given a set of $m$ images as validation test, for each test in the suite:
\begin{enumerate}
\item Let $I$ the $i$-th analyzed image, a detection map $D_i$ is computed with our module.
\item Let $\mathcal{P}_i$ the subset of pixels in the image corresponding to the spliced region and let $\mathcal{S}_{\mathcal{P}_i}$ their corresponding values extracted from $D$. Finally, let $N_i$ the subset of pixels in the image corresponding to its original part and let $\mathcal{S}_{\mathcal{N}_i}$ their corresponding values extracted from $D_i$.
\item Step 1 and 2 are repeated for each image in the validation step, joining the resulting sets of scores:
$$
\mathcal{S}_{\mathcal{P}} = \cup_{i = 1}^{m} \mathcal{S}_{\mathcal{P}_i}
$$
$$
\mathcal{S}_{\mathcal{N}} = \cup_{i = 1}^{m} \mathcal{S}_{\mathcal{N}_i}
$$
\end{enumerate}

The two resulting sets of scores, $\mathcal{S}_{\mathcal{P}}$ and $\mathcal{S}_{\mathcal{N}}$ are used for evaluating overall performances.

The size of the evaluation dataset is limited to 25 image due the  required computational time. In the classification stage, we used an SVM classifier with an RBF kernel, and classical grid search for adjusting parameters in training samples \cite{bishop2007pattern}.

Experimental results are collected in Table \ref{table:performanceregionaldet}.

\begin{table}[h!]
\centering
\begin{tabular}{l c c c c c} 
\hline \hline 
\textbf{Test case} & \textbf{Train} & \textbf{RC} & \textbf{ACC} & \textbf{AUC} &\textbf{ F-Score} \\ [0.5ex]
\hline
Test 1 & - & Median & 0.49 & 0.32 & 0.25\\
Test 2 & - & Global & 0.52 & 0.40 & 0.27\\
Test 3 & SplicedCC & Median & 0.54 & 0.53 & 0.26\\
Test 4 & SplicedCC & Global & 0.57 & 0.57 & 0.31\\
Test 5 &	 SplicedDSO & Median & 0.53 & 0.50 & 0.27\\
Test 6 &	 SplicedDSO & Global & 0.61 & 0.63 & 0.33\\ [1ex]
\hline
\end{tabular}
\caption{Performance of region forgery detection module.}
\label{table:performanceregionaldet}
\end{table}

\subsection{Test cases}

\subsubsection{Performance without training}

\begin{figure}[!htb]
\minipage{0.42\textwidth}
  \includegraphics[width=\linewidth]{train_cc/roc_normal}
\endminipage\hfill
\minipage{0.53\textwidth}
  \includegraphics[width=\linewidth]{train_cc/prec_rec_normal}
\endminipage
\caption{ROC curve and Precision-Recall curve for classification without training, using the median reference value.}\label{fig:regiondetnormal}
\end{figure}

\begin{figure}[!htb]
\minipage{0.42\textwidth}
  \includegraphics[width=\linewidth]{train_cc/roc_normal_global}
\endminipage\hfill
\minipage{0.53\textwidth}
  \includegraphics[width=\linewidth]{train_cc/prec_rec_normal_global}
\endminipage
\caption{ROC curve and Precision-Recall curve for classification without training, using the global reference value.}\label{fig:regiondetnormal}
\end{figure}

\subsubsection{Performance with SVM training over \emph{SplicedCC}}

\begin{figure}[!htb]
\minipage{0.42\textwidth}
  \includegraphics[width=\linewidth]{train_cc/roc_svm}
\endminipage\hfill
\minipage{0.53\textwidth}
  \includegraphics[width=\linewidth]{train_cc/prec_rec_svm}
\endminipage
\caption{ROC curve and Precision-Recall curve for classification with SVM training on \emph{SplicedCC}, using the global reference value.}\label{fig:regiondetnormal}
\end{figure}

\begin{figure}[!htb]
\minipage{0.42\textwidth}
  \includegraphics[width=\linewidth]{train_cc/roc_svm_global}
\endminipage\hfill
\minipage{0.53\textwidth}
  \includegraphics[width=\linewidth]{train_cc/prec_rec_svm_global}
\endminipage
\caption{ROC curve and Precision-Recall curve for classification with SVM training on \emph{SplicedCC}, using the global reference value.}\label{fig:regiondetnormal}
\end{figure}

\subsubsection{Performance with SVM training over \emph{SplicedDSO}}

\begin{figure}[!htb]
\minipage{0.42\textwidth}
  \includegraphics[width=\linewidth]{train_dso/roc_svm}
\endminipage\hfill
\minipage{0.53\textwidth}
  \includegraphics[width=\linewidth]{train_dso/prec_rec_svm}
\endminipage
\caption{ROC curve and Precision-Recall curve for classification with SVM training on \emph{SplicedDSO}, using the global reference value.}\label{fig:regiondetnormal}
\end{figure}

\begin{figure}[!htb]
\minipage{0.42\textwidth}
  \includegraphics[width=\linewidth]{train_dso/roc_svm_global}
\endminipage\hfill
\minipage{0.53\textwidth}
  \includegraphics[width=\linewidth]{train_dso/prec_rec_svm_global}
\endminipage
\caption{ROC curve and Precision-Recall curve for classification with SVM training on \emph{SplicedDSO}, using the global reference value.}\label{fig:regiondetnormal}
\end{figure}


\chapter*{Conclusions}
\addcontentsline{toc}{chapter}{Conclusions}
\chaptermark{Conclusions}

Image composition is among the most common types of image manipulation and consists of using parts of two or more images to create a new fake one. In this context, this work has presented a method, composed by two different modules, that relies on illumination inconsistencies for detecting this image forgeries.

In the two different approaches, divided into two modules, proposed in Chapter 2, we analyzed illuminant maps entailing the interaction between the light source and the objects contained in a scene. 

The first module, aimed at detecting forgeries involving people, is based on the assumption that similar materials (\emph{e.g.} the human skin) illuminated by a common light source have similar properties in such maps. Features based on color of illuminant maps are now used to describe human face regions and a set of kNN models are trained and used for classification using a fusion technique. However, although image composition involving people is one of the most usual ways of modifying images, other elements can also be inserted into them. To address this issue, we combine the first module with another kind of approach, still based on the illuminant maps analysis, but completely image content independent.

As for research directions and future work, we suggest different contributions for each one of our proposed approaches. Future developments of this work may include the extension of the first module considering additional and different parts of body (\emph{e.g.}, all skin spots of the human body visible in an image). Given that our method compares skin material, it is feasible to use additional body parts, such as arms and legs, to increase the detection and confidence of the method.

Although promising as forensic evidence, methods that operate on illuminant color are inherently prone to estimation errors. Thus, we expect that further improvements can be achieved when more advanced illuminant color estimators become available.
}

\appendix
\chapter{Screenshots delle pagine Web}

\begin{figure}[h]
\centering
\includegraphics[width= 14cm]{homepage.jpg}
\caption{\textit{Homepage}}\label{homepage}
\end{figure}

\begin{figure}[h]
\centering
\includegraphics[width= 14cm]{results.jpg}
\caption{\textit{Risultati della ricerca}}\label{results}
\end{figure}

\begin{figure}[h]
\centering
\includegraphics[width= 14cm]{my-collection.jpg}
\caption{\textit{Collezione personale}}\label{my-collection}
\end{figure}

\begin{figure}[h]
\centering
\includegraphics[width= 14cm]{video-component.jpg}
\caption{\textit{Componente per video}}\label{video-component}
\end{figure}

\begin{figure}[h]
\centering
\includegraphics[width= 14cm]{image-component.jpg}
\caption{\textit{Componente per immagini}}\label{image-component}
\end{figure}

\begin{figure}[h]
\centering
\includegraphics[width= 14cm]{audio-component.jpg}
\caption{\textit{Componente per audio}}\label{audio-component}
\end{figure}

\begin{figure}[h]
\centering
\includegraphics[width= 14cm]{document-component.jpg}
\caption{\textit{Componente per documenti}}\label{document-component}
\end{figure}

\begin{figure}[h]
\centering
\includegraphics[width= 14cm]{upload-page.jpg}
\caption{\textit{Pagina di upload media}}\label{upload-page}
\end{figure}

\begin{figure}[h]
\centering
\includegraphics[width= 14cm]{edit-profile-page.jpg}
\caption{\textit{Pagina modifica profilo personale}}\label{edit-profile-page}
\end{figure}

%%%  BACK MATTER  %%%  (bibliografia)
\backmatter

{
\bibliographystyle{ieee}
\bibliography{bibliography}
}

%\chapter*{Acknowlegments}
\addcontentsline{toc}{chapter}{Acknowlegments}
\chaptermark{Acknowlegments}

\end{document}

%pagina bianca (da usare, nei punti giusti, x avere una pag bianca invece che vuota con intestazione)
%\newpage{ \thispagestyle{empty}\null\vfil}