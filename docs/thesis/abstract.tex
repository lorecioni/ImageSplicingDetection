\chapter*{Abstract}
\addcontentsline{toc}{chapter}{Abstract}
\chaptermark{Abstract}

With the advent of powerful image editing tools, manipulating images and changing their content is becoming a trivial task. Unfortunately, most of the times, these modifications are aimed deceiving viewers, changing opinions or even affecting how people perceive reality. Therefore, the development of efficient and effective forgery detection techniques has become a hot research topic. 

From all types of image forgeries, image compositions are specially interesting. This type of forgery uses parts of two or more images to construct a \emph{new reality} from scenes that never happened. Among all different telltales investigated for detecting image compositions, illumination inconsistencies are considered the most promising since a perfect light matching in a forged image is still difficult to achieve. 

This thesis builds upon the hypothesis that image illumination inconsistencies are strong and powerful evidence of image composition and presents an extension of two approaches in literature, both based on the illuminant maps analysis. The first method is specific for human face forgeries, the second is a blind approach for detecting regional splicing.


\chapter*{Sommario}
\addcontentsline{toc}{chapter}{Sommario}
\chaptermark{Sommario}

Con l'avvento di sempre più potenti strumenti di modifica di immagini, la loro manipolazione è diventata un'operazione molto comune. Purtroppo, il più delle volte, queste modifiche sono volte ad ingannare gli spettatori, cambiare le loro opinioni o perfino a modificare il modo in cui percepiscono la realtà. Di conseguenza, lo sviluppo di tecniche sempre più efficienti ed efficaci per la rilevazione di fotomontaggi  è diventato un argomento di ricerca molto importante. 


Tra tutti i tipi di falsificazione, la composizione di più immagini è particolarmente interessante. Questo tipo di operazione utilizza le parti di due o più immagini al fine di costruire una vera e propria nuova realtà, ricostruire scene mai accadute realmente. Tra tutte le diverse tecniche di rilevazione di composizione di immagini, quelle basate sulle inconsistenze della luce sono considerate fra le più promettenti, poichè è quasi sempre difficile riuscire a rispettare la coerenza di illuminazione all'interno della stessa immagine quando questa viene modificata.

Questa tesi pone le sue basi proprio sull'ipotesi che l'individuazione di inconsistenze nella luce presenti in un'immagine sia una prova evidente di contraffazione. A partire da questa ipotesi, è presentata un'analisi ed un'estensione di due approcci presenti in letteratura che pongono le loro basi sulle \emph{Illuminant Maps}. Il primo metodo è specifico per la rilevazione di contraffazioni che riguardano volti umani nelle immagini; il secondo invece è un approccio molto più generico che cerca di individuare zone dell'immagine che presentano delle inconsistenze nel colore dell'illuminante.