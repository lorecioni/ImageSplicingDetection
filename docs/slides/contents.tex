%-- Intro --%

\begin{tframe}{Introduction}

\vspace{0.5cm}
Digital images are easy to manipulate thanks to the availability of the \textbf{powerful editing software} and \textbf{sophisticated digital cameras}.

\vspace{1cm}

\begin{minipage}{\textwidth}
\begin{columns}[T]
\begin{column}{0.5\textwidth}
\vspace{0.1cm}
The development of methods for verifying \textbf{image authenticity} is a real need in forensics.

\vspace{0.8cm}
\textbf{Purpose}: to detect image splicing  aimed at \emph{deceiving} the viewer.
\end{column}
\begin{column}{0.4\textwidth}
\includegraphics[width=0.8\textwidth]{images/image-editing.jpg}
\end{column}
\end{columns}
\end{minipage}

\end{tframe}

%-- Image compositions --%

\begin{tframe}{Forgery detection approaches}

\vspace{0.2cm}
Image forensic detection techniques search for \textit{traces} that can be grouped into:
\vspace{0.3cm}
\begin{enumerate}
\item \textbf{\textit{Signal} level}: signal specific properties, called \emph{footprints}, left during the editing phase that can be revealed using signal processing-based tools.
\vspace{0.3cm}
\item \textbf{\textit{Scene} level}: exploiting inconsistencies in scene shadows, lights, reflections, perspective, and geometry of objects. Main advantage: being fairly independent on low-level characteristics of images, they are extremely robust to compression, altering, and other image processing operations
\end{enumerate}
\vspace{0.3cm}
%
%One of the main advantages of these techniques is that, being fairly independent on low-level characteristics of images, they are extremely robust to com- pression, �ltering, and other image processing operations, remaining applicable even when the quality of the image is low.
\end{tframe}

%-- Light based detection --%

\begin{tframe}{Lighting-based inconsistencies}
Methods based on \textbf{lighting inconsistencies} are particularly \emph{robust}: a perfect illumination adjustment in a image composition is very hard to achieve.
\begin{center}
\includegraphics[width=0.4\textwidth]{images/lighting-based.jpg}
\end{center}
\begin{enumerate}
\item \textbf{Object light source inconsistencies}
\vspace{0.1cm}
\item \textbf{Illuminant colors inconsistencies}
\vspace{0.1cm}
\begin{enumerate}
\item \textit{Specular dichromatic reflectance models} [5]
\vspace{0.1cm}
\item \textit{Illuminant Maps (IMs)}
\end{enumerate}
\end{enumerate}
\vspace{0.3cm}
\end{tframe}


\begin{tframe}{Illuminant Maps estimation}
\vspace{0.2cm}
For the \emph{Illuminant Maps} estimation, two different \emph{state-of-art} techniques are used: 
\vspace{0.3cm}
\begin{enumerate}
\item A \emph{statistical-based} approach using \textbf{Generalized Grayworld Estimate (GGE)} algorithm [2]. Rely on hypotheses related to statistics of image pixels (e.g.  the \emph{gray world assumption} [6]).
\vspace{0.2cm}
\item A \emph{physics-based} approach using \textbf{Inverse-Intensity Chromaticity (IIC)} method [4]. Rely on theoretical formulations of how light interacts with objects (e.g. \emph{the dichromatic reflectance model})
\end{enumerate}

\begin{center}
\includegraphics[width=0.55\textwidth]{images/riess.jpg}
\end{center}

\end{tframe}



\begin{tframe}{Proposed approach}
The current \emph{state of the art} approaches requires human interaction. 
$$\Downarrow$$
\begin{center}
\textbf{Main goal}: make the approach user independent.
\end{center}
Two different starting points:
\vspace{0.1cm}
\begin{itemize}
\item \textbf{Face forgery detection module}: specifically for detecting forgeries involving people. Based on the work presented by Carvalho et al. [1]. Improving and automating the detection process.

\item \textbf{Regional forgery detection module}: image content independent. Based on the work presented by Fan et al. [2]. A more general and experimental approach.
\end{itemize}

\end{tframe}

\begin{tframe}{Face forgery detection module - 1}

\begin{center}
\includegraphics[width=1\textwidth]{images/pipeline_faces.jpg}
\end{center}
Introducing a \emph{face detector} and \emph{soft classification scores}.
\end{tframe}


\begin{tframe}{Face forgery detection module - 2}
\vspace{0.1cm}
The output of the algorithm consist in a \textbf{forgery score} for the image and a score for each detected face.

\vspace{0.2cm}
\begin{center}
\includegraphics[width=0.6\textwidth]{images/facedetectionoutput.jpg}
\end{center}
\end{tframe}


\begin{tframe}{Regional forgery detection module - 1}
\begin{center}
\includegraphics[width=1\textwidth]{images/pipeline_regions.jpg}
\end{center}
\vspace{0.2cm}
Two different considered \textbf{reference colors} are considered:
\vspace{0.3cm}
\begin{itemize}
\item \emph{Median}: the median illuminant color for each direction
\vspace{0.1cm}
\item \emph{Global}: the whole image illuminant color
\end{itemize}
\end{tframe}

\begin{tframe}{Regional forgery detection module - 2}
\vspace{0.1cm}
The output of the algorithm consist in a \textbf{forgery score} for the image and the forgery \textbf{regional mask}.
\vspace{0.2cm}
\begin{center}
\includegraphics[width=1\textwidth]{images/regionalresult.jpg}
\end{center}

\vspace{0.2cm}
Work best in presence of \textbf{uniform backgrounds colors} and simple light settings.
\end{tframe}



\begin{tframe}{Evaluation datasets}
\begin{itemize}
\item \textbf{DSO-1}: 200 indoor and outdoor images (100 original and 100 doctored) with an image resolution of 2048 x 1536 pixels.
\vspace{0.1cm}
\item \textbf{DSI-1}: 50 downloaded images (25 original and 25 doctored) with different resolutions. Original images are downloaded from \emph{Flickr}, doctored images collected from different websites.
\end{itemize}
\begin{figure}[!htb]
\minipage{0.5\textwidth}
\centering
  \includegraphics[width=0.6\linewidth]{images/dso_sample_spliced}
  \caption{DSO-1 sample spliced image}\label{fig:dsoimage}
\endminipage\hfill
\minipage{0.5\textwidth}
\centering
  \includegraphics[width=0.7\linewidth]{images/dsi_sample_spliced}
  \caption{DSI-1 sample spliced image}\label{fig:dsiimage}
\endminipage
\end{figure}

\end{tframe}

\begin{tframe}{Experimental results - 1}
Experimental results for face forgery detection module.
\begin{footnotesize}
\begin{table}[h!]
\centering
\begin{tabular}{l c c c c c} 
\hline \hline 
\textbf{Test case} & \textbf{Train} & \textbf{Test} & \textbf{Accuracy} & \textbf{AUC} &\textbf{ F-Score} \\ [0.5ex]
\hline
Test 1 & DSO-1 & DSO-1 &	0.84 & 0.90	& 0.78\\
Test 2 & DSI-1 & DSI-1 &	0.89 & 0.92 & 0.89\\
Test 3 &	DSO-1 &	DSI-1 &	0.59 & 0.58 & 0.64\\
Test 4 &	DSI-1 & DSO-1 & 0.63 & 0.60 & 0.54\\ [1ex]
\hline
\end{tabular}
\caption{Performance of face forgery detection module over paired faces using non-uniform weights.}
\end{table}
\end{footnotesize}
\vspace{0.2cm}
Good results in cross-validation evaluations. In a cross dataset approach, the dataset used for training makes the difference.
\end{tframe}


\begin{tframe}{Experimental results - 2}
Experimental results for regional forgery detection module. 
\begin{footnotesize}
\begin{table}[h!]
\centering
\begin{tabular}{l c c c c c} 
\hline \hline 
\textbf{Test case} & \textbf{Train} & \textbf{RC} & \textbf{ACC} & \textbf{AUC} &\textbf{ F-Score} \\ [0.5ex]
\hline
Test 1 & - & Median & 0.49 & 0.32 & 0.25\\
Test 2 & - & Global & 0.52 & 0.40 & 0.27\\
Test 3 & SplicedCC & Median & 0.54 & 0.53 & 0.26\\
Test 4 & SplicedCC & Global & 0.57 & 0.57 & 0.31\\
Test 5 &	 SplicedDSO & Median & 0.53 & 0.50 & 0.27\\
Test 6 &	 SplicedDSO & Global & 0.61 & 0.63 & 0.33\\ [1ex]
\hline
\end{tabular}
\caption{Performance of region forgery detection module}
\label{table:performanceregionaldet}
\end{table}
\end{footnotesize}
Better results achieved using the\textbf{ global illuminant color }as reference.\\

\vspace{0.2cm}
\textbf{Very low accuracy}: not suitable for a forensic approach.
\end{tframe}

\begin{tframe}{Conclusions}
\begin{itemize}
\vspace{0.1cm}
\item Two different approaches for forgery detection are presented: a face forgery detection module and a generic region forgery detection module.
\vspace{0.1cm}
\item Face module achieved most promising results, but it works only with images involving people forgeries.
\vspace{0.1cm}
\item \textbf{Future developments}: given that our method compares skin material, it is feasible to use additional body parts, such as arms and legs, to increase the detection and confidence of the method.
\vspace{0.1cm}
\item Further improvements can be achieved when more advanced illuminant color estimators become available.
\end{itemize}
\end{tframe}
